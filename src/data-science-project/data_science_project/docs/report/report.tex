\documentclass{article}
\usepackage[utf8]{inputenc}



\title{Grupo Bimbo Inventory Demand \\Data Science Report }
\author{Juan Manuel Serrano Rodriguez, Nicolas Guevara Herran,\\ Giovanny Esteban Moreno Rondon}
\date{July 2024}


\begin{document}

\maketitle

\section{Gathering Data and Exploration}
Laureado: Isamu Akasaki (Japón), Hiroshi Amano (Japón), Shuji Nakamura (Japón, Estados Unidos de América).\\
País: Japón y Estados Unidos de América
\subsection{Kaggle and Web Scrapping}
Nació el 11 de septiembre de 1960 en Hamamatsu, (Japón). Doctorado en ingeniería en 1989 por la Universidad de Nagoya, donde es catedrático. Graduado en 1983 del Departamento de Electrónica de la Facultad de Ingeniería por la Universidad de Nagoya. En 1982, se unió al grupo de Isamu Akasaki. Investigador asociado de la Universidad de Nagoya de 1988 a 1992, asistente, asociado y catedrático de la Universidad Meijo entre 1992 y 2010 y catedrático de la Universidad de Nagoya. Galardonado con el Premio Nobel 2014 de Física, junto con Isamu Akasaki y Shuji Nakamura por su invención de las luces azules LED.
\subsection{Data Exploration}
Nació el 30 de enero 1929 en Chiran, Japón. En 1952, se graduó de la Facultad de Ciencias por la Universidad de Kioto y comenzó a trabajar en la Kobe Kogyo Corporation, que forma parte de Fujitsu. En 1959, ingresó a la Universidad de Nagoya como investigador asociado en la Facultad de Ingeniería y como profesor asistente a partir de 1964 tras doctorarse. Fue reconocido por inventar el nitruro de galio brillante (GaN) azul unión p-n LED en 1989 y, posteriormente, el alto brillo azul GaN LED. Comenzó a trabajar en GaN-based blue LEDs en la década de 1960 en el Matsushita Research Institute de Tokio, Inc. (MRIT). Nombrado en 1981 catedrático de la Facultad de Ingeniería de la Universidad de Nagoya, puesto en el que se retiró en 1992.

\section{Data Preprocessing}
Invención de diodos emisores de luz azul que han permitido el desarrollo de fuentes de luz blanca y brillante de bajo consumo.\\\\
1) Su invención ha revolucionado la iluminación de las dos últimas décadas al permitir generar una luz blanca, brillante y barata.\\
2) Permitió crear lámparas led blancas, que emiten una luz brillante, son de larga duración y alta eficiencia energética

\section{Feature Engineering}
Los científicos japoneses Isamu Akasaki, Hiroshi Amano y Shuji Nakamura han conseguido el premio Nobel de Física 2014 por haber inventado el led azul, una nueva fuente de luz eficiente, duradera y amigable con el medioambiente. 
Su invento permitió crear lámparas led blancas, que emiten una luz brillante, son de larga duración y alta eficiencia energética. Constantemente están mejorando, con mayores flujos luminosos (medidos en lúmenes) por unidad de energía eléctrica de entrada (medido en vatios). El registro más reciente es poco más de 300 lm/ W, en comparación con los 16 de las bombillas regulares y los cerca de 70 de las lámparas fluorescentes.
\section{Model Selection}
Cuando Akasaki, Amano y Nakamura produjeron haces brillantes de luz azul en semiconductores a principio de la década de 1990, desencadenaron una transformación fundamental en la tecnología de iluminación. Los diodos verdes y rojos ya se conocían desde hacía tiempo, pero sin el componte azul, las lámparas blancas no se podían crear. La invención del led azul tiene solo 20 años, pero ya ha contribuido a crear luz blanca de una forma nueva beneficiándonos a todos.

\section{Model Training}
El premio nobel otorgado a los autores mencionados anteriormente ha sido para dar reconocimiento a la gran labor que ellos hicieron por la creación de un nuevo invento de bajo consumo. La invención del led azul tiene solo 20 años, pero ya ha contribuido a crear luz blanca de una forma nueva beneficiándonos a todos.



\section{Model Evaluation}
ecuadoruniversitario.com  (2014). Quito – Ecuador. Isamu Akasaki, Hiroshi Amano y Shuji Nakamura reciben el Premio Nobel de Física 2014.  Recuperado de: http://ecuadoruniversitario.com/ciencia-y-tecnologia/isamu-akasaki-hiroshi-amano-y-shuji-nakamura-reciben-el-premio-nobel-de-fisica-2014/\\

\section{Conclusions}
ecuadoruniversitario.com  (2014). Quito – Ecuador. Isamu Akasaki, Hiroshi Amano y Shuji Nakamura reciben el Premio Nobel de Física 2014.  Recuperado de: http://ecuadoruniversitario.com/ciencia-y-tecnologia/isamu-akasaki-hiroshi-amano-y-shuji-nakamura-reciben-el-premio-nobel-de-fisica-2014/\\

SINC. (2014). Madrid – España. Premio Nobel de Física 2014 para los creadores del led azul. Recuperado de: https://www.agenciasinc.es/Noticias/Premio-Nobel-de-Fisica-2014-para-los-creadores-del-led-azul 














\end{document}
